\documentclass[a4paper, 14pt]{extarticle}
\usepackage[russian]{babel}
\usepackage[T1]{fontenc}
\usepackage{fontspec}
\usepackage{indentfirst}
\usepackage{enumitem}
\usepackage[
  left=20mm,
  right=10mm,
  top=20mm,
  bottom=20mm
]{geometry}
\usepackage{parskip}
\usepackage{titlesec}
\usepackage{xurl}
\usepackage{hyperref}
\usepackage{longtable}
\usepackage{array}
\usepackage{diagbox}

\newcolumntype{L}[1]{>{\raggedright\let\newline\\\arraybackslash\hspace{0pt}}m{#1}}
\newcolumntype{C}[1]{>{\centering\let\newline\\\arraybackslash\hspace{0pt}}m{#1}}
\newcolumntype{R}[1]{>{\raggedleft\let\newline\\\arraybackslash\hspace{0pt}}m{#1}}

\hypersetup{
  colorlinks=true,
  linkcolor=black,
  filecolor=blue,
  urlcolor=blue,
}

\renewcommand*{\labelitemi}{---}
\linespread{1.5}
\setmainfont{PT Astra Serif}

\renewcommand{\baselinestretch}{1.5}
\setlength{\parindent}{1.25cm}
\setlength{\parskip}{6pt}

\setlength{\parindent}{1.25cm}
\setlist[itemize]{itemsep=0em,topsep=0em,parsep=0em,partopsep=0em,leftmargin=1.55\parindent}
\setlist[enumerate]{itemsep=0em,topsep=0em,parsep=0em,partopsep=0em,leftmargin=1.55\parindent}

\renewcommand{\thesection}{\arabic{section}.}
\renewcommand{\thesubsection}{\thesection\arabic{subsection}.}
\renewcommand{\thesubsubsection}{\thesubsection\arabic{subsubsection}.}

\titleformat{\section}
{\normalfont\bfseries}{\thesection}{0.5em}{}

\begin{document}

\begin{flushright}
  \textit{Швалов Даниил К33211}
\end{flushright}

\begin{center}
  \bfseries
  Домашнее задание №3

  <<Цифровые инструменты для удаленной работы>>
\end{center}

Для определения наиболее подходящего инструмента удаленной работы необходимо
выделить набор критериев, по которым можно понять то, насколько тот или иной
инструмент подходит для команды. Для себя я выделил следующие пять критериев,
которыми я обычно руководствуюсь при выборе инструмента удаленной работы.

\textbf{Доступность для всех основных устройств}. В наше время существует
множество различных видов устройств, а также операционных систем. Чаще всего в
одной команде у разных участников могут быть разные операционные системы,
например, Linux, Windows или MacOS, а также Android или iOS. Поскольку
инструмент для удаленной работы может использоваться по несколько раз на дню,
он должен поддерживаться всеми платформами, которыми пользуются члены команды.
В противном случае трудно гарантировать стабильную и комфортную работу всех
участников команды в удаленном режиме.

\textbf{Стоимость}. Одним из важных критериев инструмента для удаленной работы
является его стоимость. Не каждая команда может позволить себе оплатить
дорогостоящий инструмент для проведения онлайн встреч. Особенно критична
стоимость для совсем небольших команд, например, стартапов, которые весьма
ограничены в своих средствах. Поэтому низкая стоимость или гибкая модель оплаты
может стать решающим фактором для команды.

\textbf{Управление ролями}. В целях безопасности, а также удобства
взаимодействия, команде может потребоваться наличие возможности разграничения
ролей и прав доступа в инструменте для удаленной работы. Это может быть
особенно полезно для больших команд, в которых разные участники имеют различные
сферы ответственности. Кроме того, некоторые встречи должны быть
необщедоступными, то есть не каждый участник по тем или иным причинам должен
иметь возможность присоединиться к встрече. Другими словами, инструмент,
поддерживающий изолированные комнаты с разграничением прав доступа, в некоторых
случаях будет выигрышно смотреться на фоне конкурентов.

\textbf{Планирование встреч}. Чаще всего встреча не возникает внезапно, а
планируется заранее. Инструмент, поддерживающий планирование встреч, сильно
упростит жизнь команде, поскольку каждому из участников будет сразу понятно
когда, где и во сколько состоится та или иная встреча.

\textbf{Запись встречи}. По различным причинам некоторые участники команды не
могут присутствовать на некоторых важных встречах команды. Исправить эту
ситуацию частично может функционал записи встреч. Запись встречи позволяет
участникам команд, которые по каким-либо причинам пропустили встречу, понять
контекст, проблему, возможные решения и прочие аспекты встречи, которые были
обговорены во время отсутствия этого участника.

В качестве инструментов для удаленной работы будут рассмотрены такие приложения
как Zoom, Discord и Google Meet. Сравнение данных инструментов по ранее
приведенным критериям представлены в следующей таблице.

{
\renewcommand*{\arraystretch}{1.5}
\begin{longtable}{|C{0.28\textwidth}|C{0.21\textwidth}|C{0.21\textwidth}|C{0.21\textwidth}|}
  \hline
  \backslashbox[0.30\textwidth]{\textbf{Критерий}}{\textbf{Приложение}} & \textbf{Zoom}                                                                                        & \textbf{Google Meet}                                                                              & \textbf{Discord}                                                                                                    \\
  \hline
  \textbf{Доступность для всех основных устройств}                      & Да                                                                                                   & Да                                                                                                & Да                                                                                                                  \\
  \hline
  \textbf{Стоимость}                                                    & Доступна бесплатная и платная версия                                                                 & Доступна бесплатная и платная версия                                                              & Доступна бесплатная и платная версия                                                                                \\
  \hline
  \textbf{Управление ролями}                                            & Ограниченный набор ролей (организатор, соорганизатор, участник) с фиксированным набором возможностей & Ограниченный набор ролей (организатор, участник) с возможностью настройки возможностей участников & Гибкая модель управления ролями. Возможность назначать каждому участнику специфичную роль с разными правами доступа \\
  \hline
  \textbf{Планирование встреч}                                          & Да                                                                                                   & Да                                                                                                & Нет                                                                                                                 \\
  \hline
  \textbf{Запись встречи}                                               & Да                                                                                                   & Доступно платной версии                                                                           & Нет                                                                                                                 \\
  \hline
\end{longtable}
}

Исходя из приведенного сравнения можно понять, что все три инструмента, по
моему мнению, могут подойти большинству команд. Выбор конкретного инструмента
из данных трех будет сильно зависеть от специфики и предпочтения команды.

\end{document}
