\documentclass[a4paper, 14pt]{extarticle}
\usepackage[russian]{babel}
\usepackage[T1]{fontenc}
\usepackage{fontspec}
\usepackage{indentfirst}
\usepackage{enumitem}
\usepackage[
  left=20mm,
  right=10mm,
  top=20mm,
  bottom=20mm
]{geometry}
\usepackage{parskip}
\usepackage{titlesec}
\usepackage{xurl}
\usepackage{hyperref}

\hypersetup{
  colorlinks=true,
  linkcolor=black,
  filecolor=blue,
  urlcolor=blue,
}

\renewcommand*{\labelitemi}{---}
\linespread{1.5}
\setmainfont{PT Astra Serif}

\renewcommand{\baselinestretch}{1.5}
\setlength{\parindent}{1.25cm}
\setlength{\parskip}{6pt}

\setlength{\parindent}{1.25cm}
\setlist[itemize]{itemsep=0em,topsep=0em,parsep=0em,partopsep=0em,leftmargin=1.55\parindent}
\setlist[enumerate]{itemsep=0em,topsep=0em,parsep=0em,partopsep=0em,leftmargin=1.55\parindent}

\renewcommand{\thesection}{\arabic{section}.}
\renewcommand{\thesubsection}{\thesection\arabic{subsection}.}
\renewcommand{\thesubsubsection}{\thesubsection\arabic{subsubsection}.}

\titleformat{\section}
{\normalfont\bfseries}{\thesection}{0.5em}{}

\begin{document}

\begin{flushright}
  \textit{Швалов Даниил К33211}
\end{flushright}

\begin{center}
  \bfseries
  Домашнее задание №1

  <<Особенности работы в удаленной команде>>
\end{center}

В наше время очень популярна работа в удаленном формате. На примере компании X5
Group я бы хотел рассмотреть сильные и слабые стороны удаленного формата
работы, а также сложности и подводные камни перехода.

X5 Group — российская розничная торговая компания, управляющая продуктовыми
торговыми сетями такими как «Пятерочка», «Перекресток» и др. До 2020 года
большинство сотрудников компании работали в очном формате. Весной 2020 года
ситуация сильно изменилась из-за коронавирусной инфекции. Из-за этого большую
часть сотрудников пришлось перевести на дистанционный формат.

При переводе сотрудников с офисов на дистант компания столкнулась с множеством
крупных и мелких проблем. Для себя я выделил три основные сложности, которые,
на мой взгляд, стали ключевыми при переходе.

Первой сложностью, с которой столкнулась компания, был выбор инструмента
совместной работы. Большинство известных сервисов не подходили с точки зрения
политик безопасности, в том числе и <<Zoom>>. В итоге в качестве инструментов
были выбраны <<Microsoft Teams>> как решение для видеоконференций, совместной
работы и отслеживания задач, и <<Skype>> в качестве альтернативы для быстрого
обмена сообщениями и созвонов.

Второй сложностью при переходе на удаленку стала закупка техники и софта. На
момент начала пандемии не у всех сотрудников компании было необходимое
оборудование, поэтому его нужно было закупить.

Третьей сложностью стали психологические проблемы из-за постоянной работы из
дома. Многим не хватало личного общения с коллегами, возможности собраться и
провести время вместе. Чтобы решить эту проблему, компания организовала ряд
активностей: утренние зарядки, онлайн-викторины и т.п.

Несмотря на все сложности, дистанционный формат показал свои сильные стороны.
Он предоставил компании и людям гибкость в работе: сотрудники не тратят время
на дорогу и более эффективны в рабочее время, повысилась скорость выполнения
задач. У сотрудников появилась возможность уделять больше времени семье и своим
увлечениям. В целом дистанционный формат экономит время, деньги и силы. Также
удаленная работа сделала географические ограничения гораздо менее значимыми.

Однако не всем сотрудникам комфортно работать полностью на удаленке, многим
необходим личный контакт для некоторых задач. Оптимальным для компании оказался
гибридный формат. Так, например, практика показала, что виртуальные переговоры
оказались очень удобны — в них можно быстро собирать встречи и обмениваться
файлами тем, кто находится в офисе, и тем, кто подключается по сети. Поэтому
такие созвоны плотно вошли в жизнь в компании. Также офисы были переделаны в
трансформируемые пространства для совместной работы как в очном, так и в
дистанционном формате, что добавило гибкости и удобства для сотрудников.

Опираясь на личный опыт, а также на опыт компании X5 Group, я считаю, что
лучшее решение --- это гибридный формат. На мой взгляд, он сочетает в себе все
преимущества очного и дистанционного форматов, при этом практически нивелируя
недостатки каждого из них. Мы все разные и по-разному предпочитаем работать,
порой в совершенно разных условиях. Как известно, эффективность труда сильно
зависит от условий труда, поэтому гибридный формат позволяет сделать так, чтобы
большей части сотрудников было максимально комфортно. Кроме того, гибридный
формат может сэкономить время и деньги не только компании, но и сотрудников,
что также позитивно сказывается и на тех, и на других.

\end{document}
